In this chapter, I analyze the assumptions made by three email services --- senders, receivers, and forwarders --- that
represent services developed by three different parties. I  demonstrate how these assumptions do not always hold in practice, leading to security
vulnerabilities. I show how these vulnerabilities can be exploited to bypass email authentication
mechanisms, allowing attackers to impersonate legitimate senders and forge emails.

\section{Introduction}
% Email is an essential communication tool today.
Email has long been a uniquely popular medium for social engineering
attacks.\footnote{In the 2021 Verizon Data Breach Investigation
  Report, phishing is implicated in 36\% of the more than 4,000 data
  breaches investigated; and email-based attacks, including Business
  Email Compromise (BEC), completely dominate the social engineering
  attack vector~\cite{dbir2021}.}  While it is widely used for
both unsolicited business correspondence as well as person-to-person
communications, email provides no intrinsic integrity guarantees.  In
particular, the baseline SMTP protocol provides no mechanism to
establish if the purported sender of an email message (\eg, From:
\dns{Anthony.Blinken@state.gov}) is in fact genuine.

To help address this issue, starting in the early 2000's, the email
operations community introduced multiple anti-spoofing protocols,
including the Sender Policy Framework (SPF)~\cite{rfc7208}, DomainKeys
Identified Mail
(DKIM)~\cite{rfc6376} and Domain-based Message Authentication
Reporting and Conformance (DMARC)~\cite{rfc7489}, each designed to
tighten controls on which parties can successfully deliver
mail purporting to originate from particular domain names.  However,
these protocols had the disadvantage of being both post-hoc (needing
to support existing email deployments and conventions) and piecemeal
(each addressing slightly different threats in slightly different
ways).  As a result, the composition of these protocols is complex and
hard to reason about, leading to a structure that Chen \etal\ recently
demonstrated can enable a range of evasion
attacks~\cite{chen2020composition}.

In this Chapter, I explore the unique aspects of this problem created
as a result of \emph{email forwarding}, which is commonly used by both
individuals (\ie, to aggregate mail from multiple accounts) and
organizations (\ie, for mailing list distribution).  While clearly
useful, forwarding introduces a range of new interaction
complexities. First, forwarding involves three parties instead of two
(the sender, the forwarder, and the receiver), where the
``authenticity'' of an email message is commonly determined by the party with
the weakest security settings.  Second, the intrinsic nature of email
forwarding is to transparently send an existing message to a new
address ``on behalf'' of its original recipient --- a goal very much
at odds with the anti-spoofing function of protocols such as SPF and
DMARC.  For this reason, forwarded email messages can receive special
treatment based on various assumptions about how forwarding is used in
practice.  Finally, there is no single standard implementation of
email forwarding. Different providers make different choices and the
email ecosystem is forced to accommodate them.  Unfortunately, some
problematic implementation choices (\eg, permitting ``open
forwarding'') incur no security impact on the implementing party but
can jeopardize the security of downstream recipients.  This inversion of
incentives and capabilities creates additional challenges to
mitigating forwarding vulnerabilities.

To characterize the nature of these issues, I conduct a large-scale empirical
measurement study to infer and characterize the mail forwarding
behaviors of 16 leading email providers and four popular mailing
list services.  From these results, I identify a range of implicit
assumptions and vulnerable features in the
configuration of senders, receivers, and forwarders.  Using a
combination of these factors, I then demonstrate a series of distinct
evasion attacks that bypass existing anti-spoofing protocols and allow
the successful delivery of email with spoofed sender addresses (\eg,
From: \dns{Anthony.Blinken@state.gov}).  These attacks affect both leading
online email service providers (\eg, Gmail, Microsoft Outlook, iCloud, and
Zoho) and mailing list providers/software (\eg, Google Groups and
Gaggle).  Moreover, some of these issues have extremely broad
impact --- affecting the integrity of email sent from tens of
thousands of domains, including those representing organizations in the
US government (spanning the majority of US cabinet domains, such as
\dns{state.gov} and \dns{doe.gov}, as well as the domains of security agencies
such as \dns{odni.gov}, \dns{cisa.gov}, and \dns{secretservice.gov}), financial services
(\eg, \dns{transunion.com}, \dns{mastercard.com}, and \dns{discover.com}), news (\eg,
\dns{washingtonpost.com},
\dns{latimes.com},
\dns{apnews.com}, and \dns{afp.com}), commerce (\eg, \dns{unilever.com}, \dns{dow.com}), and law (\eg,
\dns{perkinscoie.com}).
Finally, in addition to disclosing these issues to their respective
providers, I discuss the complexities involved in identifying,
mitigating, and fixing such problems going forward.

% \alex{I know this is a very lame way to write a paper. Summarizing the results again to make sure I are on the same page. Free feel to delete it.}
% In summary, I present the following main contributions:
% \begin{itemize}
%   \item I conduct a large-scale empirical measurement study to infer and characterize the forwarding mechanisms and behaviors of 20 prominent email forwarding services.
%   \item From these measurement results, I identify four different forwarding mechanisms used in the wild (Section~\ref{sec:measure_forwarding_mechs_and_arc}), two of which are overlooked in prior work~\cite{shen2020weak,wang2022revisiting}, leading them to underestimate the security issues with email forwarding.
%   \item I present a systematic analysis of vulnerable forwarding assumptions and features (Section~\ref{sec:assumptions}), and show how an adversary can exploit them to perform a range of email spoofing attacks (Section~\ref{sec:attacks}). The attacks I uncover affect both users of leading email service providers such as Gmail, Microsoft Outlook, iCloud, and Zoho, as well as critical domains used by government organizations and financial services among others.
% \end{itemize}
