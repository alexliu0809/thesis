\section{Ethics and Disclosure}
\label{sec:disclosure}
When sending spoofed email messages in our experiments,
we took deliberate steps to avoid impacting any real users.
First, I only sent spoofed email messages to accounts that I created ourselves.
Second, I initially tested each attack by spoofing domains that I created and controlled for this research.
Once I established that our attacks could succeed using these test domains,
we ran a small set of experiments that spoofed email from real domains (to validate the absence of any unforeseen protection);
however, these email messages were only sent to our test accounts and did not spoof existing, legitimate email addresses from these domains.
Finally, all of our email messages contained innocuous text (e.g., ``a spoofed email'') that would not themselves cause harm.
% even if a message had somehow been misdelivered,
% all of our messages contained innocuous content and thus was unlikely
% to cause harm.

I have disclosed all of the vulnerabilities and attacks to the
affected providers. As of the time of publication, I have received
affirmative feedback from all affected providers and I summarize our
current understanding of their present state here. Zoho has not only
patched the issue with their ARC implementation (also confirmed by
Wang et al.~\cite{wang2022revisiting}, who conducted their
measurements after the patch) and awarded us a bug bounty, but is also
further augmenting the security of its ARC implementation. Microsoft
confirmed the vulnerabilities (with severity ``Important'', the
highest severity assigned to email spoofing bugs) and awarded us a bug
bounty. They have partially fixed the issues by rejecting spoofed email
messages purporting to be from domains that have a DMARC policy of
\textsc{Reject}~\cite{hotmailreject}.
Gaggle confirmed the issues I flagged and stated that they would
start enforcing DMARC. Gmail fixed the issues I reported.  iCloud
partially fixed the issues I reported by not forwarding email
messages that fail DMARC authentication (except for domains with DMARC
policy \textsc{None}). Hushmail fixed the issues I reported by not
forwarding email messages that fail DMARC authentication. Freemail
fixed the issues I reported by not forwarding spoofed email messages
from domains that are their customers. Mail2World attempted to fix the
issues by using spam filters and remains vulnerable. Runbox did not
view the issues I reported as vulnerabilities. Instead, they consider
monitoring account activities post-complaints sufficient.
% I will report on the full set of disclosure feedback and outcomes in the final version of this paper.
