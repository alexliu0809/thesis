%-------------------------------------------------------------------------------
\section{Introduction}
%-------------------------------------------------------------------------------

Consumer mobile spyware --- software that covertly gathers information
on a mobile device and transfers that information to a remote server
--- has existed for at least two decades, but has grown significantly
in popularity in recent years.  In one recent study from Norton Labs
~\cite{AYearAft87:online}, the number of devices identified with
spyware apps increased by 63\% between September 2020 and May 2021. A
similar report from Avast saw a 93\% increase in the use of spyware
apps in the UK over a similar period~\cite{UseofSta91:online}.

Sold under a wide variety of brand names --- TheTruthSpy, mSpy,
Flexispy and so on --- these apps are marketed directly to the general
public.  They are relatively cheap (typically between \$30 and \$100
per month), easy to install and do not require specialized technical
know-how to deploy or operate.  Indeed, the only requirements for such
software is \emph{temporary} physical access to the target device
and the ability to install an off-store app.\footnote{Because their features
  violate store policy, app stores like Google Play do not permit the sale of popular spyware apps.}  After
installation, the owner of the target device may have no knowledge
that anything has changed.  But the intimate details of their life can
now be sent to another party, including the contents of their text
messages, email messages, photos taken and received, and even live
recording from their microphone and camera.  Unsurprisingly, such
``stalkerware'' has been implicated in a range of abuses including
intimate partner violence~\cite{chatterjee2018spyware} and
cyberstalking~\cite{woodlock2017abuse}.

Moreover, privacy failures in the ``back end'' software used to store and
display exfiltrated data means that exposure of the victims' private information is not limited to just the abuser who installs the software, but also to miscreants who exploit the insecure design and/or implementations of these apps.
Indeed, a plethora of reports indicate that a broad range of consumer spyware cloud services have been breached,
exposing hundreds of thousands (if not millions) of users' private data~\cite{HackerSt66:online,Companyt8:online,mSpybrea38:online,mSpyCybe86:online,Cerberus12:online,Stalkerw59:online,HackerSt50:online,Spywaref13:online,RetinaXa98:online,Hackercl62:online}.
% Moreover, the exposure of sensitive information is not necessarily
% limited to the party who installs such software, but can be multiplied
% by privacy failures of the ``back end'' software used to store and
% present access to exfiltrated device data.  Indeed, a plethora of reports indicate that a broad range of consumer spyware cloud services have been breached,
% exposing hundreds of thousands (if not millions) of users' private data~\cite{HackerSt66:online,Companyt8:online,mSpybrea38:online,mSpyCybe86:online,Cerberus12:online,Stalkerw59:online,HackerSt50:online,Spywaref13:online,RetinaXa98:online,Hackercl62:online}.

However, while the existence of such software, and the threat it poses,
is well documented in mass media, the technical methods that these apps use to pervasively mine and exfiltrate private data is not well understood.
How does such software hide on
the target device?  How does it acquire the contents of text messages
or of third-party applications? How do these apps monitor a victim's camera or microphone without notifying the user?  Are there a small number of
common techniques for bypassing protections or does each vendor
innovate independently?  Understanding these issues, as well as the
nature of the cloud services used to store the most sensitive data
captured from target devices, is the motivation for this work.

In particular, our paper describes a broad technical investigation
into 14 leading consumer spyware apps for Android-based smart
phones.
Specifically, we seek to answer two key questions:
\begin{itemize}
    \item How do spyware apps achieve their advertised functionalities? We focus on stealthy features that facilitate nonconsensual tracking.
    \item What are the measures taken by spyware apps to protect the data they collect?
\end{itemize}

\begin{table*}[t]
  \begin{tabular}{@{}llrlrll@{\hskip 5pt}l}
    App Name             & \multicolumn{2}{c}{Website Domain \hspace*{0.2in}\hfill\hspace*{0.1in} Ranking}  & \multicolumn{2}{c}{Portal Domain \hspace*{0.25in}\hfill\hspace*{0.1in} Ranking} & Target SDK &Package Name                                    \\
    \midrule
    mSPY                 &mspy.com                 &46k             & mspyonline.com  &220k                          &25               &core.update.framework                           \\
    Mobile-tracker-free  &mobile-tracker-free.com  &54k             &mobile-tracker-free.com  &54k                   &28                      &mobile.monitor.child2021                        \\
    Clevguard            &clevguard.com            &69k             & clevguard.com  &69k                            &28             &com.kids.pro                                    \\
    \ltgrey HoverWatch   &hoverwatch.com           &87k             &hoverwatch.com  &87k                            &28          &com.android.core.mntw                           \\
    \ltgrey Flexispy     &flexispy.com             &107k            &flexispy.com &107k                              &22          &com.fp.backup                                   \\
    \ltgrey Spyic        &spyic.com                &152k            &spyic.com  &152k                                &22         &com.sc.spyic.v3                                 \\
    Spyhuman             &spyhuman.com             &179k            & spyhuman.com  &179k                            &22              &m.mobile.control                                \\
    TheTruthSpy          &thetruthspy.com          &214k            & thetruthspy.com &214k                          &28               &com.systemservice                               \\
    iKeyMonitor          &ikeymonitor.com          &230k            &emcpanel.com  &1.1m                             &23        &com.sec...im20190419$^{*}$  \\
    \ltgrey Cerberus     &cerberusapp.com          &251k            & cerberusapp.com  &251k                         &23             &com.lsdroid.cerberus                            \\
    \ltgrey Spy24        &spy24.app                &284k            & spy24.net  &2.4m                               &29          &net.spy24.wifi                                  \\
    \ltgrey Spapp        &spappmonitoring.com      &421k            &spappmonitoring.com  &421k                      &26                  &com.monspap.alarm                               \\
    Meuspy               &meuspy.com               &485k            & meuspy.com  &485k                              &32        &br.com.sistema.aplicativo                       \\
    Highstermobile       &highstermobile.com       &590k            &evt17.com  &1.5m                                &30       &org.secure.smsgps                               \\
  \end{tabular}
  \caption{The 14 spyware apps we study, their website domain and its corresponding Tranco ranking, their portal domain and its corresponding Tranco ranking, and their APK's target SDK version and package name. Tranco rankings taken on May 5th, 2022.  Shading added to improve readability.
    \hspace*{0.05in} $^{*}$iKeyMonitor's full package name is `com.sec.android.internet.im.service.im20190419'. 
  \label{tab:apps_selected}}
\end{table*}


To address these questions empirically, we reverse engineered 14 of
the most popular consumer spyware apps. 
In analyzing their behavior
we make three primary contributions:
\begin{itemize}
  \item We performed the first comprehensive and in-depth analysis of mechanisms used by consumer spyware apps to bypass or trick system level isolation across a range of different feature categories.
  \item In addition to confirming the broad use of some techniques identified in the mobile malware literature (e.g., abusing ``accessibility'' APIs), we identified two novel abuses of Android
    APIs (new techniques of invisible camera access and hiding app icons). We also document that Android's threat model does not include abuses of their APIs that allow apps to conceal their icon.
  \item We tested spyware apps \emph{in
  situ} in a carefully monitored environment and analyzed their communications with the cloud service components.  We believe our work is also the first academic effort to document in detail a range of privacy deficiencies (e.g.,
  data sent in plaintext and cloud services with insecure direct object references (IDOR~\cite{IDOR62:online, IDORCWE:online}) for the contents of phone data).
\end{itemize}

Together, we believe that this work further sharpens the community's understanding
of consumer mobile spyware, providing guidance both for phone OS
vendors and regulators in their efforts to mitigate the use and
availability of such software.


