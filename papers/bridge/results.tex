\subsection{Analysis}
\newcommand{\bridgehash}[1]{\texttt{\zz#1\zz}}%In a perfect world, this would be changed to allow linebreaks anywhere in #1
\def\zz#1{%
 \ifx\zz#1\else
   #1\linebreak[1]\expandafter\zz
 \fi}

\begin{table*}[t!]
\centering
\footnotesize
\begin{tabular}{p{1.5cm}p{5.8cm}cp{5.8cm}p{1.5cm}}
\toprule
\textbf{Bridge\todo{format}} & \textbf{(Claimed) Deposit Transaction Hash} & \textbf{\makecell{Chain \&\\Token}} & \textbf{Withdraw Transaction Hash} & \textbf{\makecell{Chain \\\& Token}} \\
\midrule
Wormhole \#1 & \bridgehash{0x8bbb7befd198a5e90297f451fc43a9e90de083289a041c8af94116c785cf496d} &\makecell{Polygon\\0.5\\WSOL}  & \bridgehash{5AiesW9pKrZvCCJM8QPWmqx...dkadV1waPVLWfsnCMVQmyYaciLxEo} &\makecell{SOL\\650\\USDC}\\[0.2in]
% \hline
Wormhole \#2 & \bridgehash{0x55d1e486a8e2102e07fd6270a03f05bbee7b43bf27ebac97b95b98e068f6740e} &\makecell{Polygon\\4.3k\\MATIC}  & \bridgehash{0xbe81895b1c3172fd69b8d4d9bf726edfdc17083876c440f9414ff316999237d7} &\makecell{AVAX\\2.3k\\WMATIC}\\

% \cline{2-5}
%  & \bridgehash{0x542efe5a7f965b927a294bce7f3a30492d0e2ca60da2e2d6cdfaf523a28a2a9c} & \makecell{Polygon\\0.5 SOL}  &\\ 
\hline

Anyswap \#1 & \bridgehash{0x01ba4719c80b6fe911b091a7c05124b64eeece964e09c058ef8f9805daca546b} & N/A &  \bridgehash{0xf015a6b06a13a08d3499ece17504d14a95d6af3e04ae11f291dca22dbbf6c991}  & \makecell{BSC\\$5*10^{-8}$\\USDC}\\[0.2in]
% \cline{2-5}
% \cline{2-5}
Anyswap \#2 & \bridgehash{0x01ba4719c80b6fe911b091a7c05124b64eeece964e09c058ef8f9805daca546b} & N/A &  \bridgehash{0x9e55b7295880dce76aa8af0f3e3f9e36499ae0bdb28088a5924daf29c6132ceb}  & \makecell{BSC\\55k\\USDC} \\[0.2in]

% \cline{2-5}
Anyswap \#3 & \bridgehash{0xe3b0c44298fc1c149afbf4c8996fb92427ae41e4649b934ca495991b7852b855} & N/A &  \bridgehash{0x4f038804d0622d2eab15d21d902a3fdd3bdfb5427bb5fd65b9eb0a41169534be}  & \makecell{Polygon\\100k\\USDC}\\[0.2in]

Anyswap \#4 & \bridgehash{0x0x0000000000000000000000000000000000000000000000000000000000000000} & N/A &  \bridgehash{0x98aa9e94d4fd0a05c27eb13ac2e699e4426c8dd9d57d04c0fa09cf4951eb2f94}  & \makecell{BSC\\650k\\USDC}\\[0.2in]

Anyswap \#5 & \bridgehash{0x0x0000000000000000000000000000000000000000000000000000000000000000} & N/A &  \bridgehash{0xa67ac5dc308142f89409df89dc85e8fab88c575b3adef77fbc8f51858b7bf7cb}  & \makecell{Polygon\\50k\\USDC}\\[0.2in]

Anyswap \#6 & \bridgehash{0x28b233a4dbda8b4dfae7245b8fff434de95f6dbd101e1a9cb22a95ded1315a16} & \makecell{Fantom\\6.4k\\POPS} & \bridgehash{0x76bdcfd5ddfa358bf4181556e3b4f1fdd2d648a246bfab91386bdfbd7b76d01f} & \makecell{Avalanche\\6.4k\\POPS}\\[0.2in]


& \bridgehash{0xf0b5568dfd8a4559d30adc9dfc881875210a3b9dfa680d392b33eb1d2cc86cfa} & \makecell{Fantom\\6.4k\\POPS} & \bridgehash{0xc86297f14f32a33232149025d4e8f8e50985d76ac1b7ccaf181501820c0b1cf7} & \makecell{Avalanche\\6.4k\\(any)POPS}\\[0.2in]


Anyswap \#7 & \bridgehash{0x0x0000000000000000000000000000000000000000000000000000000000000000} & N/A &  \bridgehash{0xde790e8dc59d8bae7ebdf89c4b75267a6e0783219b32aebe83e112aac6c299f5}  & \makecell{Avalanche\\54k\\USDC}\\ 



\hline
HECO \#1 & \bridgehash{0x6f9d2e82aef87fc649198976974c05d4c540dacca5043ffee619cc33f3ba4cf5} & \makecell{ETH\\5m\\USDT} & \bridgehash{0x628e878fb723cf0dd838eb956ce78d23b45b130876a625fd4d283e62ac2289f0}  & \makecell{HECO\\5m\\USDT}\\[0.2in]

& \bridgehash{0x6f9d2e82aef87fc649198976974c05d4c540dacca5043ffee619cc33f3ba4cf5} & \makecell{ETH\\5m\\USDT} & \bridgehash{0x27a1e6a66b6e0fc5fa805f7400dd07397bb92226926868a82afb44154a32128b} & \makecell{HECO\\5m\\USDT}\\


\hline
Harmony \#1 & \bridgehash{0x559bc92656a6956a5ffe9eea6f14a5d5993520e31a1a08551d5171ad8f658886} & \makecell{BSC\\5.3k\\BUSD} & \bridgehash{0xdf3bf1a8227ede87d7905c026c3b6a3504cc81399ebd08e1273e1a9dd2c748a9}  & \makecell{Harmony\\5.3k\\BUSD}\\[0.2in]

& \bridgehash{0x559bc92656a6956a5ffe9eea6f14a5d5993520e31a1a08551d5171ad8f658886} & \makecell{BSC\\5.3k\\BUSD} & \bridgehash{0x304801a2b33585e6867de0c403535588979ce4d2cf41c6922223d3203589c39d} & \makecell{ETH\\$5*10^{-18}$\\BUSD}\\ 

\hline
PolyNet. \#1 & \bridgehash{0x0101010101010101010101010101010101010101010101010101010101010101} & N/A &  \bridgehash{0xd6b7f50e974311082eb4b413219f7198cbf897af4e0f2e9202b10c6afe8fa0a2}  & \makecell{ETH\\491\mil\\PLT}\\ 

\bottomrule
\end{tabular}
\caption{Other flagged transactions. PolyNet stands for Poly Network 2023. Full transaction hash for Solana: 5AiesW9pKrZvCCJM8QPWmqxsnRoQwHaQmX8NR9a8BFz3pmt2ypW67zgqeRWdkadV1waPVLWfsnCMVQmyYaciLxEo.}
\label{tab:xaction-hashes}
\end{table*}
\label{sec:retro-analysis}

%% In this section I demonstrate the effectiveness of using the
%% invariant to identify cross-chain attacks by performing a
%% retrospective analysis of known significant cross-chain attacks.

We developed a tool called \offlinetool that pairs deposit and
withdrawal transactions from blockchains into bridge transactions, and
applies the balance invariant and consistency checks on them.  Using
the transactions I collected for the 12 attacks across 11 bridges and
21 chains, \offlinetool analyzed over 10\mil bridge transactions (20\mil individual deposit and withdrawal transactions),
identifying more than 2.3\thou bridge transactions associated with the
attacks and 1.1\thou more that violated the balance invariant.

% 624+611+587+360+152+100+86+80+7.9+4.4+4.4+4.3=2621
% 3000+292+2000+642+37+336+23+3400+290+53+14=8940
% 2+18+2+1+962+15+8+16+4+136+1136+5=2305
% 3+114+15+1+2+72+2+27+15+43+1+1+2=298

Table~\ref{table:bridges-and-txns-flagged} summarizes our results.
For each attack, it shows the bridge involved, the date of the attack,
the claimed loss in USD, the number of transactions \offlinetool
analyzed, and the number and kinds of transactions that violated the
balance invariant.
%
Below I discuss these two categories of transactions in more detail.

Looking forward to Sections~\ref{sec:live-audit}
and~\ref{sec:active-protect}, altogether \offlinetool identified 3,423
bridge transactions (0.03\%) that violated the balance invariant out
of more than 10\mil bridge transactions analyzed.  If the invariant is used by a
third-party auditing or protection system, I note that such an alert
workload has a negligible overhead for manual inspection, typically
raising no more than one alert (or one batch of alerts) every few
weeks.

\subsubsection{Reported}

For each of the 12 significant cross-chain attacks,
Section~\ref{sec:retro-data} describes how I gathered the historical
transactions that correspond to the attacks using external sources.
We use these identified transactions as ground truth for evaluating
the ability of \offlinetool to identify attack transactions.  As shown
in Table~\ref{table:bridges-and-txns-flagged}, when processing the
transaction histories of the chains involved, \offlinetool
successfully identified all 2,305 bridge transactions on the source
and destination chains associated with the attacks (including a few extra ones that are missed by the public reports).

Since I designed \offlinetool to specifically identify such attacks,
these results may not be surprising.  However, they are useful for
confirming that the approach of checking a simple, well-defined
invariant works well.  Moreover, the approach works well across a
variety of models, including bridges that specify fees in a fiat
currency, tokens that use a reflection mechanism, etc.


\subsubsection{Other Violating Transactions}

%% \alex{Two questions:
%% * where do I put the comparison with prior work ()? in discussion, I can say the set of problems are broader\\
%% * cases where I are able to not alert depite the invariant being violated? in discussion, reimbursement transactions.\\
%% }

%The more compelling question
Equally compelling
is the extent to which other, non-attack
transactions violate the balance invariants.  If the
attack transactions are dominated by many false positives, then the
approach becomes less effective.
%
As shown in Table~\ref{table:bridges-and-txns-flagged}, \offlinetool
finds significantly fewer (1,117 compared to 2,305) bridge transactions that were
not previously identified as attacks.  Given the nature of these
additional transactions, though, they do not undermine the
effectiveness of the invariant approach.  By violating the invariant
something highly unusual is taking place.  As a result, I argue that
such transactions should be flagged for further scrutiny and perhaps
even blocked from executing (particularly transactions in the New category and the large transactions in
the Suspicious category below).

To ensure that I have not missed benign explanations for a violation,
we manually inspected violating bridge transactions in at least of one
of the following ways: (1) using blockchain explorers to verify that
the funds have been created by the withdrawal (and if so, whether they
have been moved); (2) searching online for any additional information
about the transaction and addresses involved; (3) checking that if
claimed deposit exists; and (4) examining unredeemed
deposits on the source chain that potentially could have been used to
back the withdrawal (e.g., because the smart contract implementation
changed between the deposit and withdrawal).  If I can manually
reconcile a bridge transaction, I consider it benign
% do not consider it an invariant violation
and do not consider it further.

We group the remaining bridge transactions that violate the invariant
into four categories, which I describe below.  For reference, I also
list some of these bridge transactions in
Table~\ref{tab:xaction-hashes} to provide specific
examples with more detail.


%% \alex{* our due diligence in verifying the unbacked transations (6.2).}

% \paragraph has a bit too much space
\newcommand{\pgraph}[1]{\vspace*{0.1in}\noindent\textbf{#1}}

\pgraph{New.}  We believe that I have identified
previously unreported bridge transactions involved in two new,
unreported attacks on Anyswap.
%% In addition to the previously reported transactions from the July 10,
%% 2021, attack on Anyswap,
The first group of three transactions executed once Anyswap reopened
after the attack but before Anyswap patched its smart contract
(Anyswap transactions \#1--3 in Table~\ref{tab:xaction-hashes}).  These transactions were three days later and involved
different deposit and withdrawal addresses than the July 10, 2021
attack (yet utilizing the same compromised key).  The second group of 21 transactions on November 18, 2021,
were all withdrawing on Avalanche.  These
transactions either referenced deposits that had already been redeemed
months previously (Anyswap \#6 in Table~\ref{tab:xaction-hashes}) or referenced non-existing deposits (Anyswap \#7 in Table~\ref{tab:xaction-hashes}).  Manually
inspecting the smart contract used, the attackers were exploiting a
bug in Anyswap's implementation (the access control code was commented
out and thus ineffective) and one of the addresses was labeled as ``KyberSwap Exploiter''. 


\pgraph{Tests.}  Two bridges had test transactions that violated the
invariant.
%
Chainswap had 15 transactions that withdrew tokens labeled as test
tokens (e.g., tokens labelled as ``testtoken'' or ``startertoken''). Similarly, Qubit had 114 transactions that minted test tokens (e.g., ``xTST'') without backing deposits.
While all these transactions did not have corresponding deposits, I surmise that they were likely benign given the tokens involved (tokens that have no real value).




\pgraph{Error.} Four bridges had transactions that suggest bugs in
either their implementation or their invocation.

PolyNetwork/2021 had one bridge transaction where the withdrawal
amount matched the deposit amount, but the withdrawal moved the funds
to the wrong destination chain (\offlinetool flagged the mismatch in
destination chain specified in the deposit and withdrawal
transactions).
%
Anyswap had two groups of bridge transactions in this category.  The
first consists of 800 withdrawal transactions that pointed to just a
few deposits. Manually sampling a few, the deposit transaction hash
was not set correctly in the withdrawal transactions (we found their
matching deposits), suggesting a program error.  The second Anyswap
group had four double-spend transactions (referencing the same
deposit) that are also likely program errors: the deposit and
withdrawal amounts are the same, and the withdrawals occurred within a
few hours of each other.
% the deposit did exist and the withdrawal was
% valid.  However, the deposit transaction hash was not set correctly in
% the withdrawal, suggesting a program error.
%
Wormhole had two bridge transactions that violated the invariant: 0.5
wrapped Solana on Polygon $\rightarrow$ 650 USDC on Solana, and
4.3\thou MATIC on Polygon $\rightarrow$ 2.3\thou on Avalanche.  The
small amounts and close proximity of the dates of the transactions
suggest they are also likely errors.

Finally, three bridges had transactions that had no apparent effect,
suggesting invocation errors or undocumented behaviors.
PolyNetwork/2023 had one bridge transaction, and Chainswap had four,
where the deposits were non-zero but the withdrawal amount was zero. And Anyswap had two withdrawals
referencing deposits that did not specify a recipient, yet the deposit
amounts covered the withdrawals.  While technically not balance
invariant violations, \offlinetool flagged them because of their
unusual circumstances.

%% These transactions effectively
%% had no effect and suggest errors by whatever invoked them or
%% undocumented fees.\alex{note they might not be considered as break the
%%   invariant (depending on our def). I also dont want to be too strong
%%   on the error part.}.



%\textbf{Harmony.}
%\geoff{thinking of  category name: sloppy, clumsy, careless, backdoor, manual}


\pgraph{Suspicious.}
%
% The nature of
The final category of bridge transactions suggests the manual,
intentional use of private keys in signing transactions that
effectively bypass verification --- precisely the kind of transactions
that warrant auditing.

%% human agency was involved in manually signing transactions that
%% otherwise would not verify.

Many of these cases involved highly suspicious unbacked bridge
transactions involving very large withdrawals without corresponding
deposits.  For example, Anyswap had seven transactions totaling more than
\$1.5\mil pointing to non-existent or already-redeemed deposits (e.g., Anyswap \#4 and \#5 in
Table~\ref{tab:xaction-hashes}).
%
PolyNetwork/2023 had 27 withdrawals referencing a non-existent
deposit address totaling more than \$20\mil (PolyNetwork/2023 \#1 in Table~\ref{tab:xaction-hashes}).

The HECO bridge had 73 unusual transactions.  One was a very
suspicious unbacked bridge transaction minting \$5\mil of USDT on the
HECO chain (HECO \#1 in Table~\ref{tab:xaction-hashes}).  The
remaining 72, totaling over \$36\mil, involved withdrawals to an
address labeled ``HECO recovery''.  The label suggests benign intent
such as rescuing funds trapped in the bridge, but the activity is also
consistent with a rug pull.

%% These
%% withdrawals are very likely recovery actions where the bridge is
%% reimbursing customers who lost funds from the attack.

Other cases suggest seemingly careless operational practices.  In particular,
%
Harmony had 43 bridge transactions that violated the invariant in a
variety of different ways.
%that make it a category unto itself.
Eight double-spending bridge transactions (e.g., Harmony \#1 in Table~\ref{tab:xaction-hashes}) used the same deposit to
release tokens twice (though only resulting in a profit of a few
hundred USD).  
Thirty-two
% bridge transactions
had indecipherable data:
it was not possible to decode the deposit function name, function
input, and some events (thus preventing verification of the deposit).
One bridge transaction minted piggybankone tokens on Harmony
chain referencing a non-existing deposit.  And two
% bridge transactions
had withdrawal amounts that were smaller than the deposits, perhaps
caused by undocumented fees or errors.
%
Considering the specific mechanism used by the Harmony bridge, where a privileged submitter key
% private key
was submitting transactions that it should not have, 
and the fact that some double-spending transactions had amounts 
different than what was deposited,
these
incidents suggest that Harmony had issues operating securely and
correctly.


%% \textbf{Anyswap.}


%% \textbf{Unbacked.}  Three bridges had unbacked bridge transactions
%% where there were withdrawals with no corresponding valid associated
%% deposits.  Some of these withdrawals are quite large, making the lack
%% of deposits suspicious.  For example,
%% %% Such transactions are highly suspicious and likely represent either
%% %% payouts or additional thefts.
%% %
%% Anyswap had two withdrawals totaling more than \$650\thou pointing to
%% non-existent deposits (Anyswap \#4 and \#5 in
%% Table~\ref{tab:xaction-hashes} in the appendix).
%% %
%% PolyNetwork had 27 withdrawals in 2023 referencing a non-existant
%% deposit address, which together totaled more than \$20\mil (Poly
%% Network/2023 \#1 and \#2 in Table~\ref{tab:xaction-hashes} are two
%% examples).

%% HECO bridge had 73 unbacked bridge transactions that fall into two
%% groups.  Most of the bridge transactions (72), totaling \geoff{\$$N$},
%% involved withdrawals to an address labeled ``HECO recovery''.  These
%% withdrawals are very likely recovery actions where the bridge is
%% reimbursing customers who lost funds from the attack.  HECO had one
%% more \geoff{much earlier?} unbacked bridge transaction minting \$5\mil
%% of USDT on the HECO chain (HECO \#1 in Table~\ref{tab:xaction-hashes})
%% \geoff{and this one is suspicious?}



% Chains Necessary for the Analysis:
% Arbitrum (poly)
% Avalanche (poly)
% missing: base
% BNB
% BSC
% ETH
% ETHPoW
% EVMOS (nomad)
% Fantom (poly)
% Gnosis (poly)
% Harmony
% HECO
% Meter
% Metis (aka Andromeda)
% Milkomeda
% missing: Moonbeam (nomad)
% Moonriver (meter)
% OKTC
% Optimism (poly)
% Polygon
% Ronin
% Solana


 


