\section{Limitations}
While our approach captures a variety of vulnerabilities, it is by no means exhaustive. For example, a key assumption I make is that withdrawal are done through designated withdraw functions. However, if an adversary is able to compromise the key to account that holds the funds for the bridge, they could simply transfer the funds to another account and then withdraw them without going through the designated withdraw function(s). Similarly, if an adversary is able to withdraw funds by repurposing other functions (e.g., the deposit function), our approach would not detect it. Moreover, if an attack transaction does not involve a withdrawal, our approach would not detect it. Last but not least, if an attack is somehow able to profit without breaking the balance invariant, our approach would not detect it.  However, I believe that our approach is a important first step in using accounting principles to protect bridges from theft and that it can be extended to address these and other vulnerabilities in the future.