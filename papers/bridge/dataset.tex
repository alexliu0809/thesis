
\begin{facingcaption}{table}
\caption[The Blockchains \offlinetool Supports and the Cross-chain Bridges that Operate on Them.]{The blockchains \offlinetool supports and the cross\-chain bridges that operate on them.  While a particular attack on a
    bridge involves two chains, I collect deposit and withdrawal
    transactions for a bridge on all chains that the bridge supports. Anyswap is also known as Multichain.}
\label{table:chain_bridge_full}
\renewcommand\tabularxcolumn[1]{>{\RaggedLeft\arraybackslash}p{#1}}
\parindent=0pt
\setbox0=\vbox{%
\vsize\textwidth
\hsize\textheight
\linewidth\hsize
\columnwidth\hsize
\textwidth\hsize
\textheight\vsize

\begin{tabular}{ll}
  \toprule
  \textbf{Blockchain} & \textbf{Bridges that Operate on the Chain} \\
  \midrule
  Arbitrum &      Anyswap, Poly Network, Wormhole \\
  Avalanche &     Anyswap, Poly Network, Meter, Nomad, Wormhole \\
  % Base &  \\ \hline
  Binance Beacon & Binance Token Hub \\
  Binance Smart  & Anyswap, Binance Token Hub, Chainswap, Harmony, Meter, Omni, Poly Network, Qubit, Wormhole \\
  Celo &          Anyswap, Poly Network, Wormhole \\
  ETH &           Anyswap, Chainswap, Harmony, HECO, Meter, Nomad, Omni, Poly Network, Qubit, Ronin, Wormhole \\
  % ETHPoW &  \\ \hline
  EVMOS &         Anyswap, Nomad, Omni \\
  Fantom &        Anyswap, Poly Network, Wormhole\\
  Gnosis &        Omni, Poly Network       \\
  Harmony &       Anyswap, Harmony Bridge, Poly Network \\
  HECO &          Anyswap, Chainswap, HECO, Poly Network \\
  Meter &         Meter\\
  Metis &         Anyswap, Poly Network\\
  Milkomeda &     Nomad \\
  Moonbeam &      Anyswap, Meter, Nomad, Wormhole\\
  Moonriver &     Anyswap, Meter \\
  OKT Chain &     Anyswap, Chainswap, Poly Network \\
  Optimism &      Anyswap, Poly Network, Wormhole\\
  Polygon &       Anyswap, Chainswap, Poly Network, Wormhole \\
  Ronin &         Ronin\\
  Solana &        Wormhole\\
  % Sui &  \\ \hline
  \bottomrule
\end{tabular}


\singlespacing
}
\centerline{\rotatebox{90}{\box0}}
\end{facingcaption}



% \begin{table*}[t]
% \centering
% \begin{tabular}{|p{2.8cm}|c|r|c|c|c|c|}\hline
%     \textbf{Name} & \textbf{Date} & \makecell{\textbf{Amount} \\ (millions)} & \makecell{\textbf{TXNs}\\ \textbf{Analyzed}} & \makecell{\textbf{Reported Malilicious} \\ \textbf{TXNs Flagged}} & \makecell{\textbf{Unreported Malilicious} \\ \textbf{TXNs Flagged}} & \makecell{\textbf{Other TXNs} \\ \textbf{Flagged}} \\ \hline
%     Ronin & 2022-03 & 624.0 & 2.8m & 2/2 & 0 & 0\\ \hline
%     PolyNetwork 2021 & 2021-08 & 611.0 & 292k& 18/18& 0 & 1\\ \hline
%     BNB Bridge & 2022-10 & 587.0 & 1.4m & 2/2 & 0& 0\\ \hline
%     Wormhole & 2022-01 & 360.0 & 500k& 1/1 & 0 & 2\\ \hline
%     Nomad & 2022-08 & 152.0 & 35k& 962/960 & 0& 0\\ \hline
%     Harmony & 2022-06 & 100.0 & 336k& 15/15& 0& 43\\ \hline
%     HECO & 2023-11 & 86.0 & 23k & 8/8 & 0 & 73\\ \hline
%     Qubit & 2022-01 & 80.0 & 260 & 16/16 & 0& 114\\ \hline
%     Anyswap & 2021-07 & 7.9 & 26k & 4/4 & 3 & 2 \\ \hline
%     PolyNetwork 2023 & 2023-06 & 4.4 & 239k & 136/136 & 0 & 28\\ \hline
%     Chainswap (07-12) & 2021-07 & 4.4 & 53k & 1136/?& 0 & 17\\ \hline
%     Meter & 2022-02 & 4.3 & 14k& 5/4 (we have 1 more)& 0& 0\\ \hline
%     % Chainswap (07-02) & 2021-07 & 4.4 & & & & \\ \hline (worth noting. still from bridge's user)
%     % Omni\alex{will be removed} & 2022-09 & 0.5 & & & Many & \\ \hline
% \end{tabular}
% \caption{List of attacks on cross-chain bridges studied, sorted by amount stolen.}
% \label{table:bridges-and-txns-flagged}
% \end{table*}

%% \begin{table*}[h]
%% \centering
%% \begin{tabular}{p{2.8cm}cr@{}l}
%%     Ronin & 2022-03 & 624 & \mil \\
%%     Ronin & 2022-03 & 4 & .4\mil \\
%% \end{tabular}
%% \label{table:bridges-and-txns-flagged2}
%% \end{table*}
\begin{table*}[t]
\scriptsize
\centering
\begin{tabular}{p{2.7cm}rS[table-format=4.2]rrcrrr}
  \toprule
%  \textbf{Name} & \textbf{Date} & \makecell{\textbf{Amount} \\ (millions)} & \makecell{\textbf{TXNs}\\ \textbf{Analyzed}} & \makecell{\textbf{Reported Malilicious} \\ \textbf{TXNs Flagged}} & \makecell{\textbf{Unreported Malilicious} \\ \textbf{TXNs Flagged}} & \makecell{\textbf{Other TXNs} \\ \textbf{Flagged}} \\
  \textbf{Name} & \multicolumn{1}{c}{\textbf{Date}} & {\textbf{Loss (USD)}} & \textbf{Analyzed} & \textbf{Reported} & \textbf{New} & \textbf{Test} & \textbf{Error} & \textbf{Suspicious} \\
  \midrule
    Ronin & Mar 2022 & \num{624.0}M\xspace & 3.0 & 2                 & - &  -  & - & - \\ % 0
    PolyNetwork/2021 & Aug 2021 & \num{611.0}M\xspace & 292\thou & 18    & - &  -  & 1 & - \\ % 1
    BSC Token Hub & Oct 2022 & \num{587.0}M\xspace & 2.0\mil & 2         & - &  -  & - & - \\ % 0
    Wormhole & Jan 2022 & \num{360.0}M\xspace & 642\thou & 1             & - &  -  & 2 & - \\ % 2
    Nomad & Aug 2022 & \num{152.0}M\xspace & 37\thou & $^\dagger$962      & - &  -  & - & - \\ % 0
    Harmony & Jun 2022 & \num{100.0}M\xspace & 336\thou & 15             & - &  -  & - & 43 \\ % 43
    HECO & Nov 2023 & \num{86.0}M\xspace & 23\thou & 8                   & - &  -  & - & 73 \\ % 73
    Qubit & Jan 2022 & \num{80.0}M\xspace & 260 & 16                     & - & 114 & - & - \\ % 114
    Anyswap & Jul 2021 & \num{7.9}M\xspace & 3.4\mil & 4                & 24 &  -  & 806 & 7 \\ % 2
    PolyNetwork/2023 & Jun 2023 & \num{4.4}M\xspace & 290\thou & 136     & - &  -  & 1 & 27 \\ %28
    Chainswap & Jul 2021 & \num{4.4}M\xspace & 53\thou & $^*$1136        & - &  15 & 4 & - \\ % 17
    Meter & Feb 2022 & \num{4.3}M\xspace & 14\thou & $^\dagger$5          & - &  -  & - & - \\ \midrule % 0
    Total  &          & \num{2.6}B\xspace & 10.1\mil & 2,305             & 24 &  129 & 814 & 150 \\
% 43+73+7+27=150
    \bottomrule
    % Chainswap (07-02) & 2021-07 & 4.4 & & & & \\ \hline (worth noting. still from bridge's user)
    % Omni\alex{will be removed} & 2022-09 & 0.5 & & & Many & \\ \hline
\end{tabular}
% 3000+292+2000+647+35+336+23+2300+240+53+14=8940
\caption[List of Top Attacks on Cross-chain Bridges in the Retrospective
  Analysis]{List of top attacks on cross-chain bridges in the retrospective
  analysis, ordered by amount stolen.  \offlinetool analyzed over
  10\mil bridge transactions (20\mil component deposit and withdrawal transactions)
  %\alex{this is pairs of transactions...i.e., the total number is 2x}
  and identified all bridge transactions previously reported as having
  been associated with the attacks (Reported).  It also identified
  bridge transactions that violated the invariant that were a
  previously unidentified attack (New), test transactions (Test),
  transactions reflecting bugs in implementations or use (Error), and
  suspicious transactions that employ manual signing (Suspicious).
  $^\dagger$\offlinetool identified slightly more transactions than
  were reported by the Nomad~\cite{nomad-groundtruth-github:online}
  and Meter~\cite{meter-groundtruth-tencent:online} bridges for their
  attacks (+2 for Nomad, +1 for Meter).
  %% A blog post mentioned 960 transactions involved in the Nomad
  %% attack, while \offlinetool identified 962.
   $^*$The Chainswap attack only has reports of the malicious deposit
  address~\cite{chainswap-groundtruth:online}, which matches the one
  identified by \offlinetool.
  %% and I report the number of violating transactions
  %% matching that address.
  %
  %% $^\diamondsuit$\offlinetool identified one more violating
  %% transaction than mentioned in online posts~\cite{}, perhaps because
  %% it involved Meter's own chain rather than \geoff{...}.
}
\label{table:bridges-and-txns-flagged}
\end{table*}

\section{Retrospective Analysis}
\label{sec:retro-results}

%% \geoff{somewhere we'll define terminology: ``bridge transaction'' $=$
%%   ``deposit transaction'' $+$ ``withdraw transaction''.  and we'll
%%   note that some malicious bridge transactions might not have a
%%   pairing (e.g., no deposit transaction).}

The basic reasoning motivating the balance invariant is simple: that
legitimate bridge transactions should conserve value.  However, this
assumes a particular model for how bridges are used and operated which
may or may not hold in practice.  To validate our hypothesis, I applied
the balance invariant analysis retrospectively across a large set of
past transactions which, while primarily benign, contain the largest
known bridge attacks during our period of study.  Our goal is both to
show that all real attacks are identified (detection), but also that
the invariant does not alert on large numbers of benign transactions
(bridge compatibility).  Later, in Sections~\ref{sec:live-audit}
and~\ref{sec:active-protect}, I will describe systems for live bridge
monitoring as a third party and a new implementation for preventing
unbalanced transactions from being committed.

%In this section I demonstrate the effectiveness of using the balance
%invariant to identify past cross-chain attacks by performing a
%retrospective analysis of known significant cross-chain attacks.
%Later in Sections~\ref{sec:live-audit} and~\ref{sec:active-protect},
%we describe systems for live bridge monitoring as a third party and a
%new implementation for protecting bridges against transactions that
%violate the invariant.

%For our retrospective analysis, I first describe how I selected the
%bridges and blockchains for our study, and how I collect bridge
%transaction data for the blockchains.  We then show that using the
%invariant correctly identifies all bridge transactions that have been
%reported for major attacks on the bridges, discuss the nature of other
%transactions that violate the balance invariant, and overall show that using
%the balance invariant to alert on suspicious bridge transactions is both
%effective and extremely low overhead.

\subsection{Data Set}
\label{sec:retro-data}

%% We start by discussing our bridge selection, followed by the blockchain selection. We end with an overview of the blockchain data collected and analyzed.

To perform a retrospective analysis I need data for which (at
least some) of the ground truth is known.   In this section we
describe how I chose the bridge attacks I study, the blockchains and
smart contracts involved, and how I collected the historical
transaction data.

%\subsubsection{Bridge Selection}
%\paragraph{Bridge Selection.}
\textbf{Attack Selection.}
%
%% To identify attacks that violate the invariant discussed in the
%% previous section,
We compiled a comprehensive list of attacks on cross-chain
bridges that occurred between January 2021 and December 2023.  We used both academic papers surveying cross-chain
bridge attacks~\cite{lee2023sok, zhang2023sok, zhao2023comprehensive} as well
as industry blog posts that collect and characterize attacks over time~\cite{GithubBridgeBugTracker, SlowMistHackedBridges:online,
  REKTDB:online, Web3Great:online, GithubBridgeHacks2:online}. 
As this chapter focuses on end-to-end auditing of bridge transactions, I filter out attacks
involving blockchains lacking smart contracts such
as Bitcoin (e.g., the pNetwork attack in 2021~\cite{pNetworkhack:online}), attacks on swap bridges (e.g., the Thor bridge attacks~\cite{Thorhack1:online,Thorhack2:online}), attacks that do not involve individual bridge transactions (e.g., the evoDefi attack~\cite{evoDefihack:online}),
and attacks that withdraw funds through other means (e.g., direct transfer from a vault that stores assets for a bridge such as the Multichain 2023 attack~\cite{Multichainhack:online}).  Moreover, given the large number of attacks during this period, I focus on
major attacks with claimed losses greater than \$1 million USD and
exclude the remaining (the sum of losses from these excluded attacks are a small fraction of those in our scope).
% where I discuss scope
% I exclude  attacks as well as swap bridges.
%% We then manually categorize each attack, analyzing blog posts that
%% describe the attack and the transactions involved.
%
%% \alex{we should also mention swap bridge out-of-scope likely somewhere
%%   discussing the invariants (something like I ignore them for the
%%   rest of the paper)} \geoff{the ``on smart contract'' requirement
%%   could also be described earlier when discussing scope}
%
This process yielded 12 attacks on 11 distinct bridges.
%, as summarized in Table~\ref{table:bridges-and-txns-flagged}.
We note that this list
includes the top five attacks on cross-chain bridges in history, which
collectively resulted in more than \$2.6 billion USD in claimed losses.

%\subsubsection{Blockchain Selection}
\textbf{Blockchain Selection.}
% The bridges involved in significant attacks 
%
Validating bridge transactions requires access to
transaction data from both the source and destination chains.  We
support every blockchain involved in the bridge transactions
associated with the 12 attacks, either as the source (or claimed
source) or destination chain.  As a result, I support a total of 21
blockchains that together cover a
broad range of designs, including many of the most popular such as 
Ethereum, Binance Smart Chain, and Solana.
% \footnote{Details about some
% deposit transactions on PolyNetwork (deposits that are used for
% liquidity purposes) are only available through PolyNetwork's API.  We
% support querying PolyNetwork's API, but do not list it as a supported
% blockchain.}
%
Table~\ref{table:chain_bridge_full} lists the blockchains and the
bridges in our retrospective study that operate on them.% \alex{cite}

%% We do not support
%% testnets, as any withdrawal transaction on mainnet that is backed by a
%% deposit on testnet should be flagged. \deian{cut testnet sentence, most people have no clue what that is}

%% \footnote{Our list of supported blockchains includes most of the
%% widely-used and well-developed blockchains. While I could explore
%% additional blockchains, the return on effort is diminishing as adding
%% support for each new blockchain is labor-intensive and is unlikely to
%% yield significant new insights.\alex{Geoff, move this paragraph
%%   around}}



%% \geoff{discuss what aspect of collecting the data requires significant
%% effort}\alex{How about a footnote}


% We then augmented the
% list of supported blockchains by including those that are popular
% among multiple bridges, have a large market cap (according to
% CoinMarketCap), and offer good API support. In total, I support 25
% blockchains.  



%\subsubsection{Smart Contract Selection}
\textbf{Smart Contract Selection.}
%
For each bridge, I comprehensively collected the results of all
versions of its bridging smart contracts on every blockchain we
considered.  In particular, I collected deposit and withdrawal
transactions created by every contract the bridge deployed on
every blockchain I support (typically many more blockchains than the
ones involved in a bridge attack) as well as all versions of the smart
contract implementation for the bridge (bridges evolve their
implementations over time, such as in response to an attack).
%
Since the transactions I collected from this set of smart contracts
are much broader than those just involved in the top 12 attacks we
consider, including them further reinforces our findings that the
alerting workload is very small (Section~\ref{sec:retro-analysis}).

%\alex{chainswap}

%% includes not only the smart contracts involved in the attacks, but
%% also other versions of the smart contracts deployed by the bridges.

%% \alex{make it clear that for every
%%   bridge, I audit every contract it has deployed on every blockchain
%%   I support, which is typically much broader than the set of the
%%   blockchains are attacked}

\begin{figure}[t]
  \centering
  \includegraphics[width=\columnwidth]{fig/sec25_plot_lifespan.pdf}
  % \caption{Timeline of bridge attacks and data collection.}
  \caption[The Lifetime of Bridges in Our Study]{The lifetime of the bridges in our retrospective study.
    Lines start with the bridge's first valid transaction and end with the
    last valid transaction in our data, corresponding to the bridge's
    closure or November 2023 if the bridge was still operating at the
    end of our data set.  Diamonds indicate the dates of attack.}
%  First valid transaction, last valid transaction, and the reported attacks on cross-chain bridges (up to Nov '23).}
  \label{fig:bridge-timeline}
\end{figure}


%\subsection{Data Collection}
%\label{sec:data-collect}
\textbf{Data Collection.}
%Since the attacks I consider involve smart contracts, the smart
%contract input and output provide verbose information for
%understanding the nature of individual bridge transactions.
Collecting smart contract transaction data---the verbose record of
what each contract did---across many chains constitutes the most
time-intensive aspect of performing a retrospective analysis.

For each bridge, I collected deposit and withdrawal transactions for
each of the blockchains it operates on (for the 21 chains we
support) using commercial RPC services which charge for queries.  We
primarily used five commercial services,\footnote{% These five RPC services are:
GetBlock, QuickNode, ChainStack, GoldRush, and Ankr.} 
as no single service supports
all of the blockchains and functionality I need.
For each bridge and blockchain combination, by default we
collected all of the historical transactions generated by its smart
contracts from the genesis of its deployment to the end of November
2023 or the end of its life, whichever came first.

The two exceptions to this rule are the Binance bridge, for which we
limited data collection to a year (six months prior to its attack and
six months after) because of the sheer volume of transactions over its
lifetime (30 million transactions) and the Harmony bridge, whose
contract was reused by another bridge (LayerZero) after its attack
(since LayerZero was not attacked itself, I only consider transactions that were directly associated with Harmony).
% \deian{what does repurposed mean?}\alex{reused}
%Since that bridge was not attacked and is outside our scope, I only
%consider transactions that are part of the Harmony bridge.


%
%% In addition, Binance Beacon Chain does not offer an easy to way filter
%% transactions, and thus I focus on deposit and withdraw transactions
%% on Binance Smart Chain.  \geoff{not sure why we're mentioning this
%%   distinction...is this an explanation for why I do not support
%%   Binance Beacon Chain?}
%
%Next, the Harmony bridge had its contract reused by another
%bridge (LayerZero) after its attack.
% \deian{what does repurposed mean?}\alex{reused}
%Since that bridge was not attacked and is outside our scope, I only
%consider transactions that are part of the Harmony bridge.

Figure~\ref{fig:bridge-timeline} illustrates these bridge lifetimes
with each line starting at a given bridge's first transaction and
ending with its last in our data set.  Diamonds indicate the dates of
individual attacks, emphasizing that many of the bridges closed
shortly after their attacks.
