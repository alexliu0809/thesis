\section{The Invariant}
\alex{Describe the invariant here (using source and dst).}
We observe that for token transfer bridges—those that facilitate the transfer of tokens between blockchains without performing token swaps—the inflow and outflow of tokens should be equivalent after factoring in fees. In other words, the outflow should be strictly less than or equal to the inflow due to the presence of fees. I refer to this principle as the invariant.


Below, I describe how the various attacks violate the invariant. For buggy deposit verification, an attacker deposits less than the amount withdrawn, resulting in a scenario where outflow exceeds inflow. For buggy event verification, compromised relaying key and compromised submitter key, the attacker withdraws assets that were never deposited, resulting in a scenario where inflow is zero and outflow is non-zero. For buggy withdraw verification, if the attacker can verify arbitrary payload, then the inflow is zero and outflow is non-zero. If the attacker can replay a message, then the inflow is non-zero and outflow is greater than inflow. In summary, in all cases, the invariant is in that the outflow is strictly greater than the inflow, rather than equal to or less than the inflow.

\alex{has been done at a grand scale}

\alex{add a figure to illustrate the invariant.}