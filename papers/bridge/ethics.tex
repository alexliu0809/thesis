\section{Ethics}
\label{sec:ethics}

We believe our work, which deals with public data, no identified
individuals and a simple means for identifying attacks on crypto token
transfer bridges (and potentially preventing such attacks in the
future), has very low ethical risk and significant upside. Moreover, I have attempted to disclose significant suspicious bridge
transactions to appropriate bridge operators (yet with limited
success as most of the bridges have ceased operations). Lastly, I will make our code and data available upon publication.
% 
% We strictly use public data and to
% the extent that I identify outcomes (e.g., attacks) they are at the
% granularity of bridges and blockchains, not individuals.

% As well,


% Considering the Menlo report principle of ``Respect for Persons'', we
% are not dealing with, nor do I attempt to analyze, data at the
% granularity of individual persons.  We strictly use public data and to
% the extent that I identify outcomes (e.g., attacks) they are at the
% granularity of bridges and blockchains, not individuals.  Similarly,
% with respect to the principle of ``Justice'', I have no reason to
% believe that our analysis discriminates at the level of persons or
% distinct groups of persons --- all users of bridges should benefit
% equally (excepting criminals).  On the topic of ``Respect for Law and
% Public Interest'', I are engaging with existing public data, I have
% documented our methods clearly and I are careful to limit our factual
% statements to those for which I have evidence.  As well, I have made
% a point to disclose significant suspicious bridge transactions to appropriate
% bridge operators.


% We believe our work, which deals with public data, no identified
% individuals and a simple means for identifying attacks on crypto token
% transfer bridges (and potentially preventing such attacks in the
% future), has very low ethical risk and significant upside.  However,
% in the interest of due diligence, I provide a more thorough
% justification of this position following the Menlo Report principles
% as indicated by the USENIX Security call.

% Considering the Menlo report principle of ``Respect for Persons'', we
% are not dealing with, nor do I attempt to analyze, data at the
% granularity of individual persons.  We strictly use public data and to
% the extent that I identify outcomes (e.g., attacks) they are at the
% granularity of bridges and blockchains, not individuals.  Similarly,
% with respect to the principle of ``Justice'', I have no reason to
% believe that our analysis discriminates at the level of persons or
% distinct groups of persons --- all users of bridges should benefit
% equally (excepting criminals).  On the topic of ``Respect for Law and
% Public Interest'', I are engaging with existing public data, I have
% documented our methods clearly and I are careful to limit our factual
% statements to those for which I have evidence.  As well, I have made
% a point to disclose significant suspicious bridge transactions to appropriate
% bridge operators.

% Finally, on the topic of ``Beneficence'', we
% identify the three sets of stakeholders in our work.  These include
% the past and (potential) future victims whose investments, both direct
% and indirect, via token transfer bridges have been undermined by
% large-scale thefts from these services.  In addition, other
% stakeholders include bridge operators and, finally, the criminal actors who
% derive income from vulnerabilities in bridges.  We argue that our
% retrospective analysis creates no direct new harm for any parties ---
% these attacks have already happened.  For bridge operators, our
% analysis may create some potential liability in supporting a civil
% claim of poor stewardship, but at the same time our documentation of
% the widespread nature of this problem provides a ``common practice''
% defense.  Some of the unilateral transfers I have documented, if they
% were illegitimate (which I do not know), could also create liability
% for bridge operators, but I believe this potential harm is balanced
% against the interest of investors in transparency and fairness in
% operating financial services.  Finally, for our proposed real-time
% analysis approaches, if these were adopted by bridge operators, it
% could foreclose a broad range of attacks.  We admit that this outcome
% could harm our criminal stakeholders as it might deny them of an
% important income stream.  Knock on effects could include a reduction
% in revenues for online crypto money laundering services and Lamborghini
% dealerships, but I believe this is well-justified by the concordant
% reduction in victimization.


