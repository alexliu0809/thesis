
\chapter{Conclusion}

This dissertation explores the problems and solutions for sharing disaggregated memory. The high
access latency and extreme cost of contention in far-memory over RDMA cause many data structures,
which would otherwise be efficient, to experience performance collapse. Stale caches cause
opportunistic data structures to fail under contention, and pointer-based data structures incur
additional round trips when far-memory pointers need to be resolved. Using existing techniques,
applications can be made to work, but they simply have to incur the penalties of sharing when under
contention.

\begin{center}
\textit{Data structures can be optimized for disaggregated memory by leveraging network programmability} \\
\end{center}


In this chapter we begin by summarizing the contributions of this dissertation and their
relationship to our thesis claim, and we conclude with a discussion of this works limitations paired
with future work directions in this area of research.

\section{Contributions}

In this dissertation, we have contributed to the state of the art in disaggregated memory systems by:

\begin{itemize}
    \item Demonstrating the significant performance gains achievable by using a programmable switch to cache the contended state of a data structure and resolve the conflicts in the network.
    \item Showing that a programmable switch can reduce the instruction-level bottlenecks of RDMA atomic operations.
    \item Demonstrating that through locality optimizations locks can be fit into a small amount of NIC memory.
    \item Showing that RDMA-verbs are well suited for locality optimized data structures.
\end{itemize}

We believe that these contributions are significant stepping stones towards the design of future
efficient disaggregated systems. In the case of {\sword}, which demonstrated acceleration on
list appends, it could be adapted to other data structures with similar properties, for example,
log-structured systems. We believe that the insight behind RCuckoo's locality-based optimizations is
general and that many data structures could benefit from localizing their data in a similar fashion.

\section{Future Work}


Each of the works presented in this dissertation is a step towards more efficient disaggregated
systems. In this section, we speculate on future directions for this area of research based on the
limitations of the work presented here.

The future for disaggregated systems is bright. This work has demonstrated that shared remote memory
can be made efficient with programmable networking hardware and that, through careful design, data
structures can be adapted to disaggregated memory. Hopefully, future work will build on these
techniques and enable a shift towards mainstream disaggregated computing.
