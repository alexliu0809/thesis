\chapter{Conclusion}
This dissertation explores high-level assumptions in service composition and their security implications. Services often make assumptions about how they will be used, which frequently do not hold in practice. Unlike low-level assumptions, which are typically expressed in the code and can be analyzed using traditional security techniques, high-level assumptions are often abstract and poorly defined. These characteristics render traditional security techniques (e.g., static analysis and machine learning) ineffective, as they typically focus on low-level code and individual components. As a result, only a limited body of work has studied the security implications of high-level assumptions in service composition, leaving significant gaps in our understanding of these assumptions.


In this thesis, I address this gap by highlighting the prevalence of vulnerable high-level assumptions in service composition. I demonstrate that services make vulnerable assumptions through case studies of three systems: email (Chapter~\ref{chap:eurosp23}), Android (Chapter~\ref{chap:pets23}), and cross-chain bridges (Chapter~\ref{chap:bridge}). In Chapter~\ref{chap:eurosp23}, I show how actions taken by forwarders can create unintended security implications, such as allowing attackers to bypass email authentication at downstream recipients. In Chapter~\ref{chap:pets23}, I demonstrate how Android spyware apps repurpose existing APIs to perform persistent and stealthy surveillance on target devices, use cases that were not intended by the API designers. In Chapter~\ref{chap:bridge}, I start by highlighting the strict security assumptions made by cross-chain bridges --- for a bridge to be secure, all of its components must be secure. I then design and implement a simple yet effective mitigation strategy that can be applied to stop a wide range of attacks --- an approach would not have been possible without a holistic understanding of the system.


\section{Future Work}
This dissertation studies high-level assumptions in service composition and their security implications. While the case studies presented in this thesis demonstrate the prevalence and severity of vulnerable high-level assumptions, several areas remain for future work. One research area could explore better approaches to identifying and analyzing assumptions in service composition. In all three cases, assumptions were identified through manual analysis, which was time-consuming and required deep domain knowledge. Future work could consider automating the process of identifying assumptions. Similarly, crafting attacks based on individual assumptions is also a manual process. A possible solution is to use fuzzing techniques to automatically generate inputs that violate assumptions. Finally, there is the general question of how to securely compose services. In this dissertation, I have shown that existing ways of composing services can have significant flaws. It remains an open question how to design protocols and mechanisms that clearly communicate assumptions, allow for flexible composition, and have strong security guarantees.

% Another area of future work 
