
\chapter{Conclusion}
This dissertation explores assumptions in service composition and their security implications. Reasoning about assumptions can be difficult, as they are often abstract, poorly defined, and require a holistic understanding of the system. These properties render traditional security techniques (e.g., static analysis and machine learning) ineffective, since they typically focus on low-level code and individual components.

In this thesis, I propose an end-to-end approach to identifying and understanding assumptions in service composition and their security implications. I highlight the effectiveness of this approach through a case study of three systems: email (Chapter~\ref{chap:eurosp23}), Android (Chapter~\ref{chap:pets23}), and cross-chain bridges (Chapter~\ref{chap:bridge}). In Chapter~\ref{chap:eurosp23}, I show how actions taken by forwarders can create unintended security implications, such as allowing attackers to bypass email authentication at downstream recipients. In Chapter~\ref{chap:pets23}, I demonstrate how Android spyware apps repurpose existing APIs to perform persistent and stealthy surveillance on a target device, use cases that were not intended by the API designers. In Chapter~\ref{chap:bridge}, I start by highlighting the strict security assumptions made by cross-chain bridges --- for a bridge to be secure, all of its components must be secure. I then design and implement a simple and effective mitigation strategy that can be applied to stop a wide range of, which would not have been possible without a holistic understanding of the system.


\section{Future Work}
This dissertation studies assumptions in service composition and their security implications. While the case studies presented in this thesis demonstrate the effectiveness of the proposed approach, there are several areas for future work. One area of research can explore better ways to identify and analyze assumptions in service composition. In all three cases, assumptions were identified through manual analysis, which was time-consuming and required deep domain knowledge. Future work could consider automating the process of identifying assumptions. Similarly, crafting attacks based on individual assumptions is also a manual process. A possible solution is using fuzzing techniques to automatically generate inputs that violate assumptions.

% Another area of future work 
