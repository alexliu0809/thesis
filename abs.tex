
\begin{dissertationabstract} 
%%	
Service composition is a common practice in modern software systems, where multiple independently developed services interact with each other using predefined protocols. This practice, while beneficial for modularity and reusability, creates a new set of security concerns --- namely, services can make inconsistent assumptions about how they interact with each other. Such assumptions are difficult to reason about, as they are often abstract, poorly defined, and require a holistic understanding of the system. These properties render traditional security techniques (e.g., static analysis and machine learning) ineffective, since they typically focus on low-level code and individual components.

In this thesis, I propose a holistic, end-to-end approach to analyzing assumptions in service composition and their security implications. Using this approach, I systematically identify the assumptions made by services in three systems: email, Android, and cross-chain bridges. I then demonstrate how these assumptions do not always hold in practice, leading to security vulnerabilities. Furthermore, I show how these vulnerabilities can be combined to create new attacks that would not be possible in isolation. I conclude by discussing and designing mitigation strategies that can address these vulnerabilities. In summary, I present an end-to-end approach for understanding security risks associated with service composition and highlight its effectiveness through three case studies.




\end{dissertationabstract}